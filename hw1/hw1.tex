\documentclass[11pt]{article}

\usepackage{../algebra}

\begin{document}

\coverpage{1}

% hw problem 1 -----------------------------------------------------------------

\begin{exercise}{7}{14}
    \problem{
        If $C$ is a finite set, let $m(C)$ denote the number of elements in $C$.
        If $A, B$ are finite sets, prove that
        $$ m(A \cup B) = m(A) + m(B) - m(A \cap B) $$
    }
    \proof{
        We take for granted the fact that $m(A \cup B) = m(A) + m(B)$ if $A,B$ are disjoint sets.
        Additionally, let's eliminate the edge case where one or both of $A,B$ is the empty set.
        Without loss of generality, let $A = \emptyset$.
        Then $m(A \cup B) = m(\emptyset \cup B) = m(B)$.
        Now on the right hand side, $m(A) + m(B) - m(A \cap B) = m(\emptyset) + m(B) - m(\emptyset) \cap B) = 0 + m(B) - m(\emptyset) = m(B) - 0 = m(B)$.
        From here on, we assume that $A,B$ are nonempty sets. \parspace
        Now let's establish some notation.
        Let $A' = A \setminus (A \cap B)$ and $B' = B \setminus (A \cap B)$.
        We then have three mutually disjoint sets $A', B',$ and $ A \cap B$.
        Also note that $A = A' \cup (A \cap B)$ and $B = B' \cup (A \cap B)$ by construction. \parspace
        We now show, via induction, that for a finite, nonempty collection of mutually disjoint, nonempty sets $ Y = \{ Y_1, Y_2, ..., Y_n \} $, we have $m(\bigcup_{i=1}^n Y_i) = \sum_{i=1}^n m(Y_i)$.
        Consider the base case where $n = 1$, then $Y = \{ Y_1 \}$.
        The left hand side becomes $m(\bigcup_{i=1}^1 Y_i) = m(Y_1)$ and the right hand side is also $\sum_{i=1}^1 m(Y_i) = m(Y_1)$ so the base case is verified. \parspace
        Now suppose that for some $k \in \N$, $m(\bigcup_{i=1}^k Y_i) = \sum_{i=1}^k m(Y_i)$.
        Now we prove that the equality holds for $k+1$. 
    }
\end{exercise}

% hw problem 2 -----------------------------------------------------------------

\begin{exercise}{7}{20}
    \problem{
        Show, for finite sets $A, B$, that $m(A \times B) = m(A)m(B)$.
    }
    \proof{
        Again, we take for granted the fact that if $A,B$ are disjoint sets then we have $m(A \cup B) = m(A) + m(B)$.
        As an edge case, consider if $A,B$, or both $A$ and $B$ are the empty set.
        In this case $A \times B = \emptyset$ and $m(A \times B) = m(\emptyset) = 0$.
        Also $m(A)m(B) = 0$ because one or both of $m(A), m(B)$ will be 0.
        Since the statement holds when one or both of $A,B$ is the empty set, we assume both sets are nonempty for the remainder of the proof. \parspace
        Now we show that for a singleton $S = \{ s \}$ and a nonempty set $X$, we have $m(S \times X) = m(X)$.
        We show this equality by establishing a bijection between $S \times X$ and $X$.
        Consider $f: S \times X \to X$ defined by $f((s, x)) = x$.
        To show that $f$ is a bijection, we must show it is surjective and injective.
        The map $f$ is surjective if every element of $X$ is the image of some element of $S \times X $ under $f$.
        For any $x' \in X$, just choose $(s, x') \in S \times X$, then $f((s, x')) = x'$.
        So $f$ is surjective, but what about injective?
        We can show $f$ is injective by proving that $f((s, x_1)) = f((s, x_2))$ implies $(s, x_1) = (s, x_2)$.
        Suppose we have $f((s, x_1)) = f((s, x_2))$, then it must also be true that $x_1 = x_2$.
        Since $s = s$ and $x_1 = x_2$, must have $(s, x_1) = (s, x_2)$.
        Therefore, $f$ is injective.
        Since $f$ is both injective and surjective, it is a bijection and $m(S \times X) = m(X)$. \parspace
        We now show, via induction, that for a finite, nonempty collection of mutually disjoint sets $ Y = \{ Y_1, Y_2, ..., Y_n \} $, we have $m(\bigcup_{i=1}^n Y_i) = \sum_{i=1}^n m(Y_i)$.
        Consider the base case where $n = 1$, then $Y = \{ Y_1 \}$.
        The left handside becomes $m(\bigcup_{i=1}^1 Y_i) = m(Y_1)$ and the right hand side is $\sum_{i=1}^1 m(Y_i) = m(Y_1)$ so the base case is verified. \parspace
        Now suppose that for some $k \in \N$, $m(\bigcup_{i=1}^k Y_i) = \sum_{i=1}^k m(Y_i)$.
        Now we prove that the equality holds for $k+1$. \parspace
        % prove the induction
        We now claim that $A \times B$ is such a finite union of sets and that those sets are mutually disjoint. \parspace
        % show set equality
        % show mutually disjoint
        % apply induction and simplify expression
    }
\end{exercise}

% hw problem 3 -----------------------------------------------------------------

\begin{exercise}{13}{6}
    \problem{
        If $f: S \to T$ is onto and $g: T \to U$ and $h: T \to U$ are such that $g \circ f = h \circ f$, then $g = h$.
    }
    \proof{
        To show that $g = h$, we must show that $g(t) = h(t)$ for all $t \in T$.
        Since $g \circ f = h \circ f$, we can say that $(g \circ f)(s) = g(f(s)) = (h \circ f)(s) = h(f(s))$ for all $s \in S$.
        Then since $f$ is onto, for every $t \in T$, there exists $s_t \in S$ such that $f(s_t) = t$.
        Then $f(s)$ can be replaced with $t$ to give $g(f(s)) = g(t) = h(f(s)) = h(t)$, which proves that $g = h$.
    }
\end{exercise}

% hw problem 4 -----------------------------------------------------------------

\begin{exercise}{20}{11}
    \problem{
        Can you find a positive integer $m$ such that $f^m = i$ for all $f \in S_4$?
    }
    \proof{
    }
\end{exercise}

% hw problem 5 -----------------------------------------------------------------

\begin{exercise}{20}{13}
    \problem{
        Show that there is a positive integer $t$ such that $f^t = i$ for all $f \in S_n$.
    }
    \proof{
    }
\end{exercise}

\end{document}
