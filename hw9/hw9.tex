\documentclass[11pt]{article}

\usepackage{../algebra}

\begin{document}

\coverpage{9}

% hw problem 1 -----------------------------------------------------------------

\begin{exercise}{91}{2}
    \problem{
        Prove that a group of order 35 is cyclic.
    }
    \proof{
        We will prove this by using Theorem 2.8.5 on page 91 of the textbook.
        Let $G$ be a group of order 35 and $p=7,q=5$ then $|G| = 35 = p*q = 7*5$ (also $p>q$).
        Then $5 \nmid (7-1) $ since $7-1=6$ and $5 \nmid 6$.
        Thus the theorem tells us that $G$ must be cyclic.
    }
\end{exercise}

% hw problem 2 -----------------------------------------------------------------

\begin{exercise}{92}{7}
    \problem{
        If $G$ is a group with subgroups $A, B$ of orders $m,n$, respectively, where $m$ and $n$ are relatively prime, prove that the subset of $G$, $AB = \{ ab \mid a \in A, b \in B \}$, has $mn$ distinct elements.
    }
    \proof{
        Note that $A \cap B \leq A,B$ which means that $|A \cap B|$ divides both $|A|=m, |B|=n$ by Lagrange's Theorem.
        But $(m,n) = 1$ so we must have $|A \cap B| = 1$.
        This means that $AB$ is an internal direct product (take $H = AB$ and $|A \cap B| = (e)$ so $H$ must be in an internal direct product).
        Since $AB$ is an internal direct product we have $AB \cong A \times B$. \parspace
        Since the two groups $AB, A \times B$ are isomorphic, there exists a bijection their underlying sets so we have $|AB| = | A \times B|$.
        Since $A,B$ are finite, $|A \times B| = mn$ so we have $|AB|=mn$ as well.
        Thus $AB$ has mn distinct elements.
    }
\end{exercise}

% hw problem 3 -----------------------------------------------------------------

\newpage
\section*{Conjugacy Stabilizers}
    \problem{
        Suppose that $K,H \leq G$ and $H$ is normal.
        Let $[a]$ stand for the $K-$conjugacy class of $a$: $[a] := \{ b a b^{-1} \mid b \in K \}$.
        Introduce the stabilizer of a: $Stab(a) := \{ b \in K \mid b a b^{-1} = a \}$. \\\\
        (a) Show that, for any $a' \in [a]$, $Stab(a')$ and $Stab(a)$ are related by conjugation, $\exists k \in K$ where
        $$Stab(a') = k \, Stab(a) \, k^{-1}$$
        Conclude that $|Stab(a)| = |Stab(a')|$. \\
        (b) Use part (a) to show that following formula for the cardinality of $[a]$:
        $$ \#[a] = \frac{|K|}{|Stab(a)|} $$
    }
    \proof{
        \\\\
        (a) Fix $a$ and $a' \in [a]$, this implies that there exists some $\bar b \in K$ such that $\bar b a \bar b ^{-1} = a'$.
        Select $k = \bar b$.
        We now wish to show that $Stab(a') = \bar b \, Stab(a) \, \bar b ^{-1}$.
        Going to the left, pick any $\bar b b \bar b ^{-1} \in \bar b \, Stab(a) \, \bar b ^{-1}$.
        Then we know that $b a b^{-1} = a$ and $\bar b a \bar b ^{-1} = a'$.
        Now consider
        $$ (\bar b b \bar b ^{-1}) a' (\bar b b \bar b^{-1})^{-1} = \bar b b \bar b^{-1}a' \bar b b^{-1} \bar b^{-1} = \bar b b a b^{-1} \bar b^{-1} = \bar b a \bar b^{-1} = a'$$
        where the simplifications were made because $b a b^{-1} = a$ and $\bar b a \bar b ^{-1} = a'$.
        So then $\bar b b \bar b^{-1}$ is also a member of $Stab(a')$. \parspace
        Now pick any $b' \in Stab(a')$ which means $b' a' (b')^{-1} = a'$.
        We wish to show that $b' \in \bar b \, Stab(a) \bar b ^{-1}$ which is true of $(\bar b ^{-1} b' \bar b) a (\bar b ^{-1} b' \bar b ^{-1})^{-1} = a$.
        So consider
        $$ (\bar b ^{-1} b' \bar b) a (\bar b ^{-1} b' \bar b ^{-1})^{-1}
        = \bar b ^{-1} b' \bar b a \bar b ^{-1} (b')^{-1} \bar b
        = \bar b ^{-1} b' a' (b')^{-1} \bar b = \bar b ^{-1} a' \bar b = a$$
        Since $\bar b a \bar b^{-1} = a' \iff \bar b^{-1} a' \bar b$.
        Thus we have $Stab(a') = \bar b \, Stab(a) \, \bar b ^{-1}$.
        This gives $|Stab(a)| = |Stab(a')|$ since the map taking $b \in Stab$ to $\bar b b \bar b ^{-1}$ is a bijection (because it has an inverse function). \parspace
        (b) We will now show that $\# [a] = \frac{|K|}{|Stab(a)|}$ by equivalently showing that $|Stab(a)| * \# [a] = |K|$.
        From the proof of part (a) we know that every $a' \in [a]$ absorbs $|Stab(a')| = |Stab(a)|$ elements of the form $b a b^{-1}$ for $b \in K$.
        Thus we have $|Stab(a)| + \cdots + |Stab(a)| = |K|$ where $\cdots$ is adding $\# [a]$ times.
        Thus we have $|Stab(a)| * \# [a] = |K| \iff \# |a| = \frac{|K|}{|Stab(a)|}$.
    }


% hw problem 4 -----------------------------------------------------------------

\newpage
\section*{Abelian Classification}
    \problem{
        List all abelian isomorphism classes with order 108.
    }
    \proof{
        Let's first find the prime factorization of 108: $108 = 54*2 = 27*2^2 = 3^3*2^2$.
        There are 3 partitions of 3: $3=3, 3=2+1, $ and $3=1+1+1$ and 2 partitions of 2: $2=2$ and $2=1+1$.
        Thus there are $2*3 =6$ nonisomorphic groups of order 108.
    }

% hw problem 5 -----------------------------------------------------------------

\begin{exercise}{101}{2}
    \problem{
        Let $G$ be an abelian group of order $p^n$, $p$ a prime, and let $a \in G$ have maximal order.
        Show that $x^{o(a)} = e$ for all $x \in G$.
    }
    \proof{
        By Lagrange's Theorem, any element in $G$ must have order $p^j$ for some $j \in \{ 0, 1, ..., n \}$.
        Let $p^k = o(a)$ the order of $a$.
        I struggled a lot on this problem so I make the following conjecture which helps me prove the problem (if true).
        Cosider the map $p^\star: G \to G$ defined by taking $x \mapsto x^p$.
        I conjecture that $p^\star$ is a homormophism onto a subgroup $H \leq G$ with order $k$.
        Since $|H| = k$ every $h \in H$ satisfies $h^k = e \in H \subset G$.
        But note that $H$ consists exactly of the elements $x^p \in G$ so $h^k = (x^p)^k = e$.
        But $x$ was arbitary in $G$ so we have shown that $x^{o(a)} = e$ for all $x \in G$.
    }
\end{exercise}


\end{document}
