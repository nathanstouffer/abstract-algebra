\documentclass[11pt]{article}

\usepackage{../algebra}

\begin{document}

\coverpage{9}

% hw problem 1 -----------------------------------------------------------------

\begin{exercise}{91}{2}
    \problem{
        Prove that a group of order 35 is cyclic.
    }
    \proof{
        We will prove this by using Theorem 2.8.5 on page 91 of the textbook.
        Let $G$ be a group of order 35 and $p=7,q=5$ then $|G| = 35 = p*q = 7*5$ (also $p>q$).
        Then $5 \nmid (7-1) $ since $7-1=6$ and $5 \nmid 6$.
        Thus the theorem tells us that $G$ must be cyclic.
    }
\end{exercise}

% hw problem 2 -----------------------------------------------------------------

\begin{exercise}{92}{7}
    \problem{
        If $G$ is a group with subgroups $A, B$ of orders $m,n$, respectively, where $m$ and $n$ are relatively prime, prove that the subset of $G$, $AB = \{ ab \mid a \in A, b \in B \}$, has $mn$ distinct elements.
    }
    \proof{
        Note that $A \cap B \leq A,B$ which means that $|A \cap B|$ divides both $|A|=m, |B|=n$ by Lagrange's Theorem.
        But $(m,n) = 1$ so we must have $|A \cap B| = 1$.
        This means that $AB$ is an internal direct product (take $H = AB$ and $|A \cap B| = (e)$ so $H$ must be in an internal direct product).
        Since $AB$ is an internal direct product we have $AB \cong A \times B$. \parspace
        Since the two groups $AB, A \times B$ are isomorphic, there exists a bijection their underlying sets so we have $|AB| = | A \times B|$.
        Since $A,B$ are finite, $|A \times B| = mn$ so we have $|AB|=mn$ as well.
        Thus $AB$ has mn distinct elements.
    }
\end{exercise}

% hw problem 3 -----------------------------------------------------------------

\newpage
\section*{Conjugacy Stabilizers}
    \problem{
        Suppose that $K,H \leq G$ and $H$ is normal.
        Let $[a]$ stand for the $K-$conjugacy class of $a$: $[a] := \{ b a b^{-1} \mid b \in K \}$.
        Introduce the stabilizer of a: $Stab(a) := \{ b \in K \mid b a b^{-1} = a \}$. \\\\
        (a) Show that, for any $a' \in [a]$, $Stab(a')$ and $Stab(a)$ are related by conjugation, $\exists k \in K$ where
        $$Stab(a') = k \, Stab(a) \, k^{-1}$$
        Conclude that $|Stab(a)| = |Stab(a')|$. \\
        (b) Use part (a) to show that following formula for the cardinality of $[a]$:
        $$ \#[a] = \frac{|K|}{|Stab(a)|} $$
    }
    \proof{
        \\\\
        (a) \parspace
        (b) We will now show that $\# [a] = \frac{|K|}{|Stab(a)|}$ by equivalently showing that $|Stab(a)| * \# [a] = |K|$.
        From part (a) we know that every $a' \in [a]$ absorbs $|Stab(a')| = |Stab(a)|$ elements of the form $b a b^{-1}$ for $b \in K$.
    }


% hw problem 4 -----------------------------------------------------------------

\newpage
\section*{Abelian Classification}
    \problem{
        List all abelian isomorphism classes with order 108.
    }
    \proof{
        Let's first find the prime factorization of 108: $108 = 54*2 = 27*2^2 = 3^3*2^2$.
        There are 3 partitions of 3: $3=3, 3=2+1, $ and $3=1+1+1$ and 2 partitions of 2: $2=2$ and $2=1+1$.
        Thus there are $2*3 =6$ nonisomorphic groups of order 108.
    }

% hw problem 5 -----------------------------------------------------------------

\begin{exercise}{101}{2}
    \problem{
        Let $G$ be an abelian group of order $p^n$, $p$ a prime, and let $a \in G$ have maximal order.
        Show that $x^{o(a)} = e$ for all $x \in G$.
    }
    \proof{

    }
\end{exercise}


\end{document}
