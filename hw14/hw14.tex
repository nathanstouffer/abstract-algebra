\documentclass[11pt]{article}

\usepackage{../algebra}

\begin{document}

\coverpage{14}

% hw problem 1 -----------------------------------------------------------------

\begin{exercise}{200}{1}
    \problem{
        Show that $a = \sqrt{2} - \sqrt{3}$ is algebraic over $\Q$ of degree at most 4 by exhibiting a polynomial $f(x)$ of degree 4 over $\Q$ such that $f(a) = 0$.
    }
    \proof{

    }
\end{exercise}


% hw problem 2 -----------------------------------------------------------------

\newpage
\section*{Field Description}
    \problem{
        Give a concrete description of teh two fields $\Q [x] / (p(x))$ and $\R [x] / (p(x))$ for $p(x) := x^2 + 2$.
        This is to say that you should identify them with some subsets of familiar fields, as done in the lecture.
    }
    \proof{

    }

% hw problem 3 -----------------------------------------------------------------

\newpage
\section*{Smallest Containing Field}
    \problem{
        Identify the smallest field containing $\Q$ and $2^{1/4}$ and show that it is a four-dimensional vector space over the field of rational numbers $\Q$.
    }
    \proof{

    }

% hw problem 4 -----------------------------------------------------------------

\newpage
\section*{Pentagon}
    \problem{
        Show that the regular pentagon (inscribed in the unit circle) is constructible by verifying that $x_0 = \sin (72 ^\circ)$ can be reached by a chain of two quadratic field extensions: $\Q \subset K \subset \Q (x_0)$.
    }
    \proof{

    }

% hw problem 5 -----------------------------------------------------------------

\newpage
\section*{Reading}
    \problem{
        Read the last paragraph on page 212.
    }
    \proof{
        Did it; and found it quite interesting!
        It seems like the final result discussed in this paragraph can be used to show that there is no general formula for roots of polynomials of degree $\geq 5$.
        I have also heard of Galois Theory being mentioned in cryptography lectures in my computer science classes, so I am interested to see the connections there.
    }

\end{document}
