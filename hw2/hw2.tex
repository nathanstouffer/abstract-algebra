\documentclass[11pt]{article}

\usepackage{../algebra}

\begin{document}

\coverpage{2}

% hw problem 1 -----------------------------------------------------------------

\begin{exercise}{47}{9}
    \problem{
        If $G$ is a group in which $a^2 = e$ for all $a \in G$, show that $G$ is abelian.
    }
    \proof{
        First note that $a^2 = e$ for all $a \in G$ implies that for every $a \in G$ $a^{-1} = a$ (for $a^{-1}$ is the element that satisifes $aa^{-1} = e$).
        Now pick any $x,y \in G$.
        Since $G$ is a group, we have $(xy)(xy)^{-1} = e$ and $(xy)^{-1} = y^{-1}x^{-1}$.
        Our assumption that $a^2 = e$ for all $a \in G$ allows us to say that $y^{-1}x^{-1} = yx$.
        Additionally, that same assumption tells us that $(xy)^{-1} = xy$.
        This gives the chain of equalities
        $$ xy = (xy)^{-1} = y^{-1}x^{-1} = yx $$
        Since $x,y$ were arbitrary members of $G$, we have just shown that $G$ is abelian.
    }
\end{exercise}

% hw problem 2 -----------------------------------------------------------------

\begin{exercise}{47}{18}
    \problem{
        If $G$ is a finite group of even order, show that there must be an element $a \neq e$ such that $a = a^{-1}$.
    }
    \proof{
        Suppose not, that is, suppose we have a finite group $G$ of even order such that $a = e$ is the only element to satisfy $a = a^{-1}$ for any $a \in G$.
        Let's define a map $f: G \to G$ which takes $x \in G$ to $x^{-1} = y \in G$.
        The map $f$ is well defined because each element in a group has an inverse.
        Note that $f$ must be a bijection ($f$ is 1-1 because inverses are unique in a group and $f$ is onto because each element in a group is an inverse).
        Further, $ff(x) = f(f(x)) = x$ because $f(x)x = yx = e$. \parspace
        Since $ff(x) = x$ for all $x \in G$, every element in $G$ is part of a 2-cycle or a 1-cycle under $f$.
        We assumed that $e$ is the only element to be its own inverse, which is equivalent to saying $e = f(e)$ is the only 1-cycle.
        Therefore, the remaining $|G|-1$ elements must be part of 2-cycles.
        But $|G|-1$ is odd so it cannot be broken into disjoint pairs.
        So we have reached a contradiction and it must be the case that some $a \in G$ ($a \neq e$) satisfies $a = a^{-1}$.
    }
\end{exercise}

% hw problem 3 -----------------------------------------------------------------

\begin{exercise}{47}{24}
    \problem{
        If $G$ is the dihedral group of order $2n$ as defined in Example 10, prove that
        \begin{enumerate}[label=(\alph*)]
            \item If $n$ is odd and $a \in G$ is such that $a \ast b = b \ast a$ for all $b \in G$, then $a = e$.
            \item If $n$ is even, show that there is an $a \in G$, $a \neq e$, such that $a \ast b = b \ast a$ for all $b \in G$.
            \item If $n$ is even, find all the elements $a \in G$ such that $a \ast b = b \ast a$ for all $b \in G$.
        \end{enumerate}
    }
    \proof{

    }
\end{exercise}

% hw problem 4 -----------------------------------------------------------------

\newpage
\section*{Custom Problem}
    \problem{
        Find the order of the group $Sym(\Q)$ of the (rigid) symmetries of the cube $Q := [-1,1]^3 \subset \R^3$.
        The task is not to just get the number but to organize all the possible symmetries and build a narrative that gets one to the answer in a rigorous way resting on few, easy to grasp geometric facts.
    }
    \proof{

    }

\end{document}
