\documentclass[11pt]{article}

\usepackage{../algebra}

\begin{document}

\coverpage{10}

% hw problem 1 -----------------------------------------------------------------

\begin{exercise}{137}{40}
    \problem{
        Prove that a finite domain is a division ring.
        As a consequence, show that $\Z _p$ is a field if $p$ is prime.
    }
    \proof{
        Let $R$ be a finite domain.
        Then for any $a,b \in R$, we know that $ab = 0$ implies that $a=0$ or $b=0$.
        Equivalently, $a,b \neq 0$ means that $ab \neq 0$.
        Now we wish to show that $R$ is a division ring.
        Since $R$ is a domain, we only must verify that $R$ contains a multiplicative identity and every non-zero element has an multiplicative inverse in $R$.
        That is we must show that $R' = R \setminus \{ 0 \}$ is a group taken with the product in $R$. \parspace
        Since $R$ is finite, take $|R| = n \leq + \infty$, which means that $|R'| = n-1$.
        Take any $r \in R'$ and consider the set of elements $\{ r, r^2, \ldots, r^n \}$.
        We know $r^k \neq 0$ for all $k$ becuase $R$ is a domain and $r \neq 0$.
        Now since $|R'| = n-1$ we must have $r^i = r^j$ for some $0 \leq i < j \leq n$.
        Let $l = j-i > 0 \implies i = j+l = l+j$ and consider
        $$ r^l r^i = r^l r^j = r^{l+j} = r^i = r^{j+l} = r^j r^l = r^i r^l $$
        Then if we take $1 = r^l$, we have a multiplicative identity: $1r^i  = r^i 1 = r^i$.
        Furthermore, consider $r^{l-1} r = r^{l-1+1} = r^l = 1$ and $r r^{l-1} = r^{1+l-1} = r^l = 1$.
        So we have an inverse as well.
        By arbitrariness of $r$, we have shown that there is an identity and inverse for each $r \in R' = R \setminus \{ 0 \}$, thus $R$ is a division ring. \parspace
        Let's think about $\Z _p$ for $p$ prime.
        We already know $\Z _p$ to a finite, commutative ring so we need only verify that $\Z _p$ is a domain.
        Then the previous result in this problem tells us that $\Z _p$ is a division ring, then commutivity gives us that the $\Z _p$ is a field.
        To check that $\Z _p$ is a domain, suppose that we have some $a,b \neq 0$ where $ab = 0$.
        Then $ab \equiv 0 \mod p \implies p \mid (ab - 0) \implies p \mid ab $ which means that the prime factorization of $ab$ must include $p$.
        But since $p$ is prime, this means that $p$ must divide either $a,b$ but this is a contradiction.
        Thus we have shown that $a,b \neq 0 \in \Z _p \implies ab \neq 0$, which is equivalent to showing that $\Z _p$ is a division ring.
    }
\end{exercise}

% hw problem 2 -----------------------------------------------------------------

\begin{exercise}{134}{10}
    \problem{
        Let $R$ be any ring with unit, and $S$ the ring of $2 \times 2$ matrices over $R$. \parspace
        {\bf (a)} Check the associative law of multiplication in $S$. \\\\
        {\bf (b)} Show that
        $T = \left\{ \left. \begin{bmatrix} a & b \\ 0 & c \end{bmatrix} \right| a,b,c \in R \right\} $ is a subring of $S$. \\\\
        {\bf (c)} Show that $\begin{bmatrix} a & b \\ 0 & c \end{bmatrix}$ as an inverse in $T$ if and only if $a$ and $c$ have inverses in $R$.
        In that case, write down $\begin{bmatrix} a & b \\ 0 & c \end{bmatrix} ^{-1}$ explicitly.
    }
    \proof{
        {\bf (a)} This amounts to just checking the equality of evaluating left to right and then right to left of three matrics in $S$.
        Let's start with left to right:
        $$
        \left(
        \begin{bmatrix}
            a & b \\
            c & d
        \end{bmatrix}
        \begin{bmatrix}
            \bar a & \bar b \\
            \bar c & \bar d
        \end{bmatrix}
        \right)
        \begin{bmatrix}
            a' & b' \\
            c' & d'
        \end{bmatrix}
        =
        \begin{bmatrix}
            a \bar a + b \bar c & a \bar b + b \bar d \\
            c \bar a + d \bar c & c \bar b + d \bar d
        \end{bmatrix}
        \begin{bmatrix}
            a' & b' \\
            c' & d'
        \end{bmatrix}
        $$
        $$
        =
        \begin{bmatrix}
            (a \bar a + b \bar c)a' + (a \bar b + b \bar d)c' & (a \bar a + b \bar c)b' + (a \bar b + b \bar d)d' \\
            (c \bar a + d \bar c)a' + (c \bar b + d \bar d)c' & (c \bar a + d \bar c)b' + (c \bar b + d \bar d)d'
        \end{bmatrix}
        =
        \begin{bmatrix}
            a \bar a a' + b \bar c a + a \bar b c' + b \bar d c' & a \bar a b' + b \bar c b' + a \bar b d' + b \bar d' \\
            c \bar a a' + d \bar c a' + c \bar b c' + d \bar d c' & c \bar a b' + d \bar c b' + c \bar b d' + d \bar d d'
        \end{bmatrix}
        $$
        and now right to left:
        $$
        \begin{bmatrix}
            a & b \\
            c & d
        \end{bmatrix}
        \left(
        \begin{bmatrix}
            \bar a & \bar b \\
            \bar c & \bar d
        \end{bmatrix}
        \begin{bmatrix}
            a' & b' \\
            c' & d'
        \end{bmatrix}
        \right)
        =
        \begin{bmatrix}
            a & b \\
            c & d
        \end{bmatrix}
        \begin{bmatrix}
            \bar a a' + \bar b c' & \bar a b' + \bar b d' \\
            \bar c a' + \bar d c' & \bar c b' + \bar d d'
        \end{bmatrix}
        $$
        $$
        =
        \begin{bmatrix}
            a(\bar a a' + \bar b c') + b(\bar c a' + \bar d c') & a(\bar a b' + \bar b d') + b(\bar c b' + \bar d d') \\
            c(\bar a a' + \bar b c') + d(\bar c a' + \bar d c') & c(\bar a b' + \bar b d') + d(\bar c b' + \bar d d')
        \end{bmatrix}
        =
        \begin{bmatrix}
            a \bar a a' + a \bar b c' + b \bar c a' + b \bar d c' & a \bar a b' + a \bar b d' + b \bar c b' + b \bar d d' \\
            c \bar a a' + c \bar b c' + d \bar c a' + b \bar d c' & c \bar a b' + c \bar b d' + d \bar c b' + d \bar d d'
        \end{bmatrix}
        $$
        Then each entry of the matrix is equal since $R$ with $+$ is an abelian group.
        \\\\
        {\bf (b)}  \\\\
        {\bf (c)}
    }
\end{exercise}

% hw problem 3 -----------------------------------------------------------------

\begin{exercise}{135}{23}
    \problem{
        Define the map $*$ in the quaternions by taking
        $$\alpha _0 + \alpha _1 + \alpha _2 + \alpha _3 \mapsto \alpha _0 - \alpha _1 - \alpha _2 - \alpha _3$$
        Then show that: \\\\
        {\bf (a)} $x^{**} = (x^*)^* = x$ \\
        {\bf (b)} $(x+y)^* = x^* + y^*$ \\
        {\bf (c)} $xx^* = x^* x$ is real an nonnegative \\
        {\bf (d)} $(xy)^* = y^* x^*$
    }
    \proof{

    }
\end{exercise}

% hw problem 4 -----------------------------------------------------------------

\begin{exercise}{135}{24}
    \problem{
        Use $*$, define $|x| = \sqrt{xx^*}$.
        Show that $|xy| = |x| |y|$ for any two quaternions $x$ and $y$, by using parts (c) and (d) of problem 23.
    }
    \proof{

    }
\end{exercise}

% hw problem 5 -----------------------------------------------------------------

\begin{exercise}{135}{25}
    \problem{
        Using the result of problem 24 to prove Lagrange's Identity.
    }
    \proof{

    }
\end{exercise}

% hw problem 6 -----------------------------------------------------------------

\newpage
\section*{Subrings of $\Q$}
    \problem{
        The rationals are our best friends.
        Let's then try to understand all subrings (with unity) of $\Q$.
        Denote by $\P$ the set of all the primes in $\N$.
        Given a subset $P \subset \P$, set
        $$ \Q _P := \{ m/n \mid \text{ prime factors of $n$ are in $P$} \} $$
        with $m/n$ being a reduce fraction: $(m,n) = 1$. \\\\
        {\bf (i)} Show that $\Q _P$ is a subring with unity of $\Q$.
        Reserve the letter $R$ for subrings with unity, $R \subset \Q$.
        Define the denominator primes associated to such rings by
        $$ P _R := \{ p \in \P \mid 1/p \in R \}$$
        {\bf (ii)} Show that if $P = P_R$ then $R = \Q _P$.
    }
    \proof{

    }


\end{document}
