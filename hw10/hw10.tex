\documentclass[11pt]{article}

\usepackage{../algebra}

\begin{document}

\coverpage{10}

% hw problem 1 -----------------------------------------------------------------

\begin{exercise}{137}{40}
    \problem{
        Prove that a finite domain is a division ring.
        As a consequence, show that $\Z _p$ is a field if $p$ is prime.
    }
    \proof{

    }
\end{exercise}

% hw problem 2 -----------------------------------------------------------------

\begin{exercise}{134}{10}
    \problem{
        Let $R$ be any ring with unit, and $S$ the ring of $2 \times 2$ matrices over $R$. \parspace
        {\bf (a)} Check the associative law of multiplication in $S$. \\\\
        {\bf (b)} Show that
        $T = \left\{ \left. \begin{bmatrix} a & b \\ 0 & c \end{bmatrix} \right| a,b,c \in R \right\} $ is a subring of $S$. \\\\
        {\bf (c)} Show that $\begin{bmatrix} a & b \\ 0 & c \end{bmatrix}$ as an inverse in $T$ if and only if $a$ and $c$ have inverses in $R$.
        In that case, write down $\begin{bmatrix} a & b \\ 0 & c \end{bmatrix} ^{-1}$ explicitly.
    }
    \proof{

    }
\end{exercise}

% hw problem 3 -----------------------------------------------------------------

\begin{exercise}{135}{23}
    \problem{
        Define the map $*$ in the quaternions by taking
        $$\alpha _0 + \alpha _1 + \alpha _2 + \alpha _3 \mapsto \alpha _0 - \alpha _1 - \alpha _2 - \alpha _3$$
        Then show that: \\\\
        {\bf (a)} $x^{**} = (x^*)^* = x$ \\
        {\bf (b)} $(x+y)^* = x^* + y^*$ \\
        {\bf (c)} $xx^* = x^* x$ is real an nonnegative \\
        {\bf (d)} $(xy)^* = y^* x^*$
    }
    \proof{

    }
\end{exercise}

% hw problem 4 -----------------------------------------------------------------

\begin{exercise}{135}{24}
    \problem{
        Use $*$, define $|x| = \sqrt{xx^*}$.
        Show that $|xy| = |x| |y|$ for any two quaternions $x$ and $y$, by using parts (c) and (d) of problem 23.
    }
    \proof{

    }
\end{exercise}

% hw problem 5 -----------------------------------------------------------------

\begin{exercise}{135}{25}
    \problem{
        Using the result of problem 24 to prove Lagrange's Identity.
    }
    \proof{

    }
\end{exercise}

% hw problem 6 -----------------------------------------------------------------

\newpage
\section*{Subrings of $\Q$}
    \problem{
        The rationals are our best friends.
        Let's then try to understand all subrings (with unity) of $\Q$.
        Denote by $\P$ the set of all the primes in $\N$.
        Given a subset $P \subset \P$, set
        $$ \Q _P := \{ m/n \mid \text{ prime factors of $n$ are in $P$} \} $$
        with $m/n$ being a reduce fraction: $(m,n) = 1$. \\\\
        {\bf (i)} Show that $\Q _P$ is a subring with unity of $\Q$.
        Reserve the letter $R$ for subrings with unity, $R \subset \Q$.
        Define the denominator primes associated to such rings by
        $$ P _R := \{ p \in \P \mid 1/p \in R \}$$
        {\bf (ii)} Show that if $P = P_R$ then $R = \Q _P$.
    }
    \proof{

    }


\end{document}
