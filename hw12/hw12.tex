\documentclass[11pt]{article}

\usepackage{../algebra}

\begin{document}

\coverpage{12}

% hw problem 1 -----------------------------------------------------------------

\begin{exercise}{139}{3}
    \problem{
        Let $p$ be an odd prime and let $1 + 1/2 + \cdots + 1/(p-1) = a/b$ where $a,b \in \Z$.
        Show that $p \mid a$.
    }
    \proof{
        Before beginning this homework, I had a busy week so I didn't get to as much of this as I wanted, sorry!
        For this problem, we gain from the hint that for any $k \in \Z _p$ we have some $n_k \in \Z$ and a unique $l_k \in \Z _p$ such that $k l_k = 1 + n_k p$ (this is gained from the fact that $(k,p) = 1$).
        Now, in the context of rational numbers, this means that $1/k = l_k / (1 + n_k p)$.
        Then our desired sum is
        $$ \sum _{k=1}^{p-1} \frac{1}{k} =
        \sum _{k=1}^{p-1} \frac{l_k}{1 + n_k p} $$
        But $l_k$ runs over all elements of $U _p$ so when we multiply the sumation by 1 in the form of the least common multiple of the terms $1 + n_k p$ divided by itself, each term in the numerator will contain a $p$ for maybe the constant terms.
        But then we have the same denominator in all the polynomials so the constant terms of each numerator sum to $-1 - 2 - \cdots - (p-1)$.
        But this is a multiple of $p$ (odd) since there are an even number of terms (so each natural number cancels and we sum a number of $p'$s).
        Thus, each term in the numerator is divisible by $p$ so the numerator is divisible by $p$.
        But then we have a rational form the sum of $1/k$ where $p$ divides the numerator as desired.
    }
\end{exercise}

% hw problem 2 -----------------------------------------------------------------

\begin{exercise}{150}{3}
    \problem{
        In example 3, show that $M = \{ x(2+i) \mid x \in R \}$ is a maximal ideal of $R$.
    }
    \proof{
        First note that $R := \{ a + ib \mid a,b \in \Z \}$ and $M := \{ x(2+i) \mid x \in R \}$.
        Now let's show that $M$ is an ideal of $R$.
        That $M$ is nonempty is satisfied because $1(2+i) = 2+i \in M$.
        $M$ is an additive subgroup of $R$ as verified by the aesthetic definition: for any $x(2+i),y(2+i) \in M$ we have that $x(2+i) - y(2+i) = (x-y)(2+i) \in M$.
        $M$ is closed by left multiplication: for any $a+bi \in R$ and $x(2+i) \in M$ we have $(a+bi)x(2+i) = x'(2+i) \in M$ where $x' = (a+bi)x'$.
        Closure from right multiplication holds since $R$ with multiplication is abelian. \parspace
        I could not find a direct proof that $M$ is maximal so I rest on the result of the previous problem that $R/M \cong \Z _5$ a field which implies that $M$ is maximal.
    }
\end{exercise}

% hw problem 3 -----------------------------------------------------------------

\begin{exercise}{150}{4}
    \problem{
        In Example 3, show that $R/M \cong \Z _5$.
    }
    \proof{
        Here we use the first isomorphism theorem.
        To do this, we must find a surjective homomorphism $\varphi$ from $R \to \Z _5$ that has $M = \ker \varphi$.
        Consider the map $\varphi : R \to \Z _5$ defined by taking $r = a+bi \mapsto a+3b$ modulo 5.
        We verify the three properties. \parspace
        Homomorphism: pick any $a+ib, a'+ib' \in R$.
        For $+$, we have $\varphi (a+ib) + \varphi (a'+ib') = a+3b \, \mod 5 + a' + 3b' \, \mod 5 = (a + a') + 3(b+b') \, \mod 5 = \varphi (a+a' + i(b+b')) = \varphi ((a+ib) + (a' + ib'))$.
        For $*$, we have
        $$ \varphi(a+ib) \varphi (a'+ib') = (a+3b \mod 5) (a'+3b' \mod 5) = aa' + 3ab' + 3a'b + 4bb' \mod 5 $$
        and
        $$ \varphi ((a+ib)(a'+ib')) = \varphi (aa' - bb' + i(ab' + a'b)) = aa' - bb' + 3(a'b + ab') \mod 5 =
        aa' + 3a'b + 3ab' + 4bb' \mod 5 $$
        where the crucial simplification steps were made with congruence/arithmetic modulu 5. \parspace
        Onto: this is checked easily, for $m \in \Z _5$ pick $m+i0 \in R$. \parspace
        Kernel: we wish to show that $\ker \varphi = M$.
        Going to the left, pick any $r = a + ib \in \ker \varphi \subset R$.
        Then $\varphi (r) = \varphi (a+ib) = 0 \mod 5 \implies a+3b \equiv 0 \mod 5 \implies a \equiv -3b \mod 5 \implies a \equiv 2b \mod 5 \iff 5 \mid (a - 2b)$ which means there exists some $k \in \Z$ such that $5k = a- 2b \iff a = 5k + 2b$.
        Then we can rewrite $r = a+bi = 5k + 2b + ib = (2+i)(2-i)k + b(2+i)$.
        Both terms are members of $M$ so their sum is a member of $M$, which is to say $r \in M$. \parspace
        Now the other direction, pick any $(a+ib)(2+i) \in M$.
        We have $(a+ib)(2+i) = 2a-b + i(a+2b) \implies \varphi (2a-b + i(a+2b))  = 2a-b + 3(a+2b) \mod 5 = 2a+3a-b+5b \mod 5 = 5a + 5b \mod 5 = 0 \mod 5$. \parspace
        So we have verified everything we need to and the first isomorphism theorem shows says that $R/M \cong \Z _5$.
    }
\end{exercise}

% hw problem 4 -----------------------------------------------------------------

\begin{exercise}{150}{5}
    \problem{
        In Example 3, show that $R / I \cong \Z _5 \oplus \Z _5$.
    }
    \proof{
        Here $R$ is the same as in the previous two problems and $I := \{ a +bi \in R \mid \, 5 \mid a \text{ and } 5 \mid b \}$.
        Note that $I$ meets all the requirements to be an ideal so it makes sense to consider $R/I$.
        For this problem, I would like to use the first isomorphism thereom again.
        Consider $\psi : R \to \Z _5 \oplus \Z _5$ defined by taking $a-ib \mapsto (a+b \mod 5, a-b \mod 5)$ (note the $-$ instead of the $+$ in the definition of $R$, we still have every member of $R$ just in a different form).
        I found that $\psi$ is a homomorphism but I failed to verify the onto and kernel properties for the first isomorphism thereorm.
        If these hold, then the two groups are isomorphic. \parspace
        Homomorphism: pick any $a-ib, x-iy \in R$.
        Addition holds by an easy check that I omit.
        Multiplication is preversed since $\psi ((a-ib)(x-iy)) = \psi (ax + by - i (ay+bx)) = (ax + by + (ay+bx), ax+by - (ay+bx)) = (a(x+y) + b(x+y), a(x-y) - b(x-y)) = (a+b,a-b)(x+y,x-y) = \varphi (a-ib) \varphi (x-iy)$ where I omit the $\mod 5$ in the co-domain to reduce clutter. \parspace
        Onto: I could not quite verify this property but here is what I have.
        For $(m,n) \in \Z _5 \oplus \Z _5$ we could require that for $a+ib$ we have $a+b = m$ and $a-b = n$.
        We know solutions for $a,b$ since the matrix $\begin{bmatrix} 1 & 1 \\ 1 & -1 \end{bmatrix}$ is invertible but our solutions for $a,b$ punch out of the integers so this does not seem like the right route.
        I look forward to the solution on this problem! \parspace
        Kernel: I also could not verify this.
    }
\end{exercise}

% hw problem 5 -----------------------------------------------------------------

\begin{exercise}{163}{1}
    \problem{
        If $F$ is a field, show that the only invertible elements in $F[x]$ are the nonzero elements of $F$.
    }
    \proof{
        We will show that an element of $p(x) \in F[x]$ is invertible if and only if $p(x)$ is nonzero in $F$.
        Going to the left, pick a $p(x) \in F[x]$ such that $p^{-1} (x)$ exists.
        That is $p * p^{-1} = p^{-1} * p = 1 \in F[x]$.
        For a contradiction, let $\deg p = n > 0$.
        This implies that $m = \deg p^{-1} > 0 $ as well since a polynomial of degree $n>0$ times a polynomial of degree 0 would still be a polynomial of degree $n$, which would not be the identity.
        But then multiplying $p$ and $p^{-1}$ gives a polynomial of degree $m+n$, which is not the identity.
        So we reached a contradction and must have that $p(x) \in F$.
        Further, it is not 0 since 0 has no multiplicative inverse. \parspace
        Now going the right, any polynomial in $f \in F[x]$ that is a nonzero element of $F$ has the inverse $f^{-1} \in F$.
    }
\end{exercise}

% hw problem 6 -----------------------------------------------------------------

\begin{exercise}{163}{3}
    \problem{
        Find the greatest common divisor of the following polynomials over $\Q$, the field of rational numbers. \\\\
        {\bf (a)} $x^3 - 6x + 7$ and $x +  4$ \\
        {\bf (b)} $x^2 - 1$ and $2x^7 - 4x^5 + 2$ \\
        {\bf (c)} $3x^2 + 1$ and $x^6 + x^4 + x + 1$ \\
        {\bf (d)} $x^3 - 1$ and $x^7 - x^4 + x^3 - 1$
    }
    \proof{
        I'm running short on time, so I didn't show much work on these ones. \\\\
        {\bf (a)} Here, $x+4$ is irreducible so we only need to test if $x+4$ divides $x^3 - 6x + 7$.
        Using long division, I found that this was not the case.
        So the greatest common divisor is the polynomial 1. \\\\
        {\bf (b)} Here $x^2 - 1 = (x+1)(x-1)$.
        Using long divion again, I found that $x-1$ divides $2x^7 - 4x^5 + 2$ but $x+1$ does not so the greatest common divisor is $x-1$. \\\\
        {\bf (c)} $3x^2 + 1$ is irreducible in $\R [x]$ so it is certainly irreducible in $\Q$.
        For this problem, I found the zeros of $3x^2 +1$ in the comple plane then computed their output in the polynomial $x^6 + x^4 + x + 1$.
        Neither resulted in 0, so $3x^2 +1$ does not divide $x^6+x^4+x+1$ and the greatest common divisor is 1. \\\\
        {\bf (d)} For this one, $x^7 - x^4 + x^3 - 1 = x^4(x^3-1) + 1(x^3-1) = (x^4+1)(x^3-1)$ so $x^3-1$ is the greatest common divisor!
    }
\end{exercise}

\end{document}
