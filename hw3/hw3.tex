\documentclass[11pt]{article}

\usepackage{../algebra}

\begin{document}

\coverpage{3}

% hw problem 1 -----------------------------------------------------------------

\begin{exercise}{48}{29}
    \problem{
        Let $G$ be a finite, nonempty set with an operation $\ast$ such that: \\
        \indent 1. $G$ is closed under $\ast$ \\
        \indent 2. $\ast$ is associative \\
        \indent 3. Given $a,b,c \in G$ with $a \ast b = a \ast c$, then $b = c$ \\
        \indent 4. Given $a,b,c \in G$ with $b \ast a = c \ast a$, then $b = c$ \\
        Prove that $G$ must be a group under $\ast$.
    }
    \proof{

    }
\end{exercise}

% hw problem 2 -----------------------------------------------------------------

\begin{exercise}{54}{3}
    \problem{
        Let $S_3$ be the symmetric group of degree 3.
        Find all the subgroups of $S_3$.
    }
    \proof{

    }
\end{exercise}

% hw problem 3 -----------------------------------------------------------------

\begin{exercise}{55}{12}
    \problem{
        Prove that a cyclic group is abelian.
    }
    \proof{

    }
\end{exercise}

% hw problem 4 -----------------------------------------------------------------

\newpage
\section*{Heisenberg group problem}
    \problem{
        Recall the general linear group $\GL _3(\R)$ of $3 \times 3$ invertible matrices with real entries (taken with the matrix product).
        Verify that the following subset, called the Heisenberg group, is a subgroup of $\GL _3 (\R)$:
        $$ \H _3 (\R) :=
        \left\{
            \begin{bmatrix}
                1 & x & z \\
                0 & 1 & y \\
                0 & 0 & 1
            \end{bmatrix}
            \mid x,y,z \in \R
        \right\} $$
    }
    \proof{

    }

% hw problem 5 -----------------------------------------------------------------

\newpage
\section*{Cube subgroups problem}
    \problem{
        Recall the group $Sym(Q)$ of the rigid symmetries of the cube $Q := [-1,1]^3$ in $\R ^3$.
        Describe in words/pictures the following: \\
        \indent a subgroup of order 4 \\
        \indent a subgroup of order 12 \\
        \indent a subgroup of order 3 \\
        \indent a subgroup of order 6 \\
        \indent a subgroup of order 8
    }
    \proof{

    }

\end{document}
