\documentclass[11pt]{article}

\usepackage{../algebra}

\begin{document}

\coverpage{3}

% hw problem 1 -----------------------------------------------------------------

\begin{exercise}{48}{29}
    \problem{
        Let $G$ be a finite, nonempty set with an operation $\ast$ such that: \\
        \indent 1. $G$ is closed under $\ast$ \\
        \indent 2. $\ast$ is associative \\
        \indent 3. Given $a,b,c \in G$ with $a \ast b = a \ast c$, then $b = c$ \\
        \indent 4. Given $a,b,c \in G$ with $b \ast a = c \ast a$, then $b = c$ \\
        Prove that $G$ must be a group under $\ast$.
    }
    \proof{
        To prove that $(G, \ast)$ is a group, we must show that $G$ contains an identity element and an inverse for each element.
        Let's begin with the identity element.
        We must find an element $e \in G$ such that $x \ast e = e \ast x = x$ for all $x \in G$.
        Let $|G| = n < + \infty$ and fix an element $g \in G$ and consider $g, g^2, g^3, ..., g^{n+1}$.
        Since $G$ is closed under $\ast$, every $g^k$ is an element of $G$, yet $G$ has only $n$ elements so we must at least have $g^i = g^j$ for some $1 \leq i < j \leq n+1$. \parspace
        Now let $l = j-i > 0$ (which means $j = l+i = i+l$) and we can say that $g^j = g^{l+i} = g^l \ast g^i$ and $g^j = g^{i+l} = g^i \ast g^l$.
        But then $g^j = g^i$ so we have $g^i = g^i \ast g^l = g^l \ast g^i$.
        Letting $g^i = \bar{g}$ and $g^l = \bar{e}$ (both elements of $G$ by closure under $\ast$) gives us $\bar{g} = \bar{g} \ast \bar{e} = \bar{e} \ast \bar{g}$ for the specific element $\bar{g} \in G$. \parspace
        We now show that $\bar{e}$ is an identity element for every element of $G$.
        Fix any $x \in G$, then $\bar{g} \ast x = \bar{g} \ast \bar{e} \ast x$ since $\bar{g} = \bar{g} \ast \bar{e}$.
        But then property 3 says that $x = \bar{e} \ast x$.
        Further, $x \ast \bar{g} = x \ast \bar{e} \ast \bar{g}$ since $\bar{e} \ast \bar{g} = \bar{g}$ and then property 4 allows us to say that $x \ast \bar{e} = x$.
        So we have just shown that $x \ast \bar{e} = \bar{e} \ast x = x$ for an arbitrary $x \in G$.
        In other words, $\bar{e}$ is an identity element for $G$. \parspace
        Now we must show an inverse element exists for every element of $G$: that there exists some element $g' \in G$ such that $g \ast g' = g' \ast g = \bar{e}$.
        To do this, we take $g$ and $g^l = \bar{e}$ as before.
        Pick $g' = g^{l-1} \in G$ then $g \ast g' = g \ast g^{l-1} = g^{1+l-1} = g^{l+1-1} = g^l = \bar{e}$ and $g' \ast g = g^{l-1} \ast g = g^{l-1+1} = g^l = \bar{e}$ as desired.
        Since we chose $g$ arbirarily, we have just shown every element has an inverse in $G$. \parspace
        So we have showed that an identity exists in $G$ and each element has an inverse so $G$ satisfies the conditions of a group.
    }
\end{exercise}

% hw problem 2 -----------------------------------------------------------------

\begin{exercise}{54}{3}
    \problem{
        Let $S_3$ be the symmetric group of degree 3.
        Find all the subgroups of $S_3$.
    }
    \proof{

    }
\end{exercise}

% hw problem 3 -----------------------------------------------------------------

\begin{exercise}{55}{12}
    \problem{
        Prove that a cyclic group is abelian.
    }
    \proof{

    }
\end{exercise}

% hw problem 4 -----------------------------------------------------------------

\newpage
\section*{Heisenberg group problem}
    \problem{
        Recall the general linear group $\GL _3(\R)$ of $3 \times 3$ invertible matrices with real entries (taken with the matrix product).
        Verify that the following subset, called the Heisenberg group, is a subgroup of $\GL _3 (\R)$:
        $$ \H _3 (\R) :=
        \left\{
            \begin{bmatrix}
                1 & x & z \\
                0 & 1 & y \\
                0 & 0 & 1
            \end{bmatrix}
            \mid x,y,z \in \R
        \right\} $$
    }
    \proof{

    }

% hw problem 5 -----------------------------------------------------------------

\newpage
\section*{Cube subgroups problem}
    \problem{
        Recall the group $Sym(Q)$ of the rigid symmetries of the cube $Q := [-1,1]^3$ in $\R ^3$.
        Describe in words/pictures the following: \\
        \indent a subgroup of order 4 \\
        \indent a subgroup of order 12 \\
        \indent a subgroup of order 3 \\
        \indent a subgroup of order 6 \\
        \indent a subgroup of order 8
    }
    \proof{

    }

\end{document}
