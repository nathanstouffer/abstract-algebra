\documentclass[11pt]{article}

\usepackage{../algebra}

\begin{document}

\coverpage{13}

% hw problem 1 -----------------------------------------------------------------

\begin{exercise}{163}{3}
    \problem{
        Find the greatest common divisor of the following polynomials over $\Q$, the field of rational numbers. \\\\
        \indent {\bf (a)} $x^3 - 6x + 7$ and $x +  4$ \\
        \indent {\bf (b)} $x^2 - 1$ and $2x^7 - 4x^5 + 2$ \\
        \indent {\bf (c)} $3x^2 + 1$ and $x^6 + x^4 + x + 1$ \\
        \indent {\bf (d)} $x^3 - 1$ and $x^7 - x^4 + x^3 - 1$
    }
    \proof{ \\\\
        {\bf (a)} Here, $x+4$ is irreducible so we only need to test if $x+4$ divides $x^3 - 6x + 7$.
        Using long division, I found that this was not the case.
        So the greatest common divisor is the polynomial 1. \\\\
        {\bf (b)} Here $x^2 - 1 = (x+1)(x-1)$.
        Using long divion again, I found that $x-1$ divides $2x^7 - 4x^5 + 2$ but $x+1$ does not so the greatest common divisor is $x-1$. \\\\
        {\bf (c)} For this problem, I found the zeros of $3x^2 +1$ in the complex plane then computed their output in the polynomial $x^6 + x^4 + x + 1$.
        Neither resulted in 0, so $3x^2 +1$ does not divide $x^6+x^4+x+1$ and the greatest common divisor is 1.
        So $3x^2 + 1$ is irreducible in $\R [x]$ so it is certainly irreducible in $\Q [x]$.\\\\
        {\bf (d)} For this one, $x^7 - x^4 + x^3 - 1 = x^4(x^3-1) + 1(x^3-1) = (x^4+1)(x^3-1)$ so $x^3-1$ is the greatest common divisor!
    }
\end{exercise}


% hw problem 2 -----------------------------------------------------------------

\begin{exercise}{164}{5}
    \problem{
        In the previous problem, let $I = \{ f(x) a(x) + g(x) b(x) \}$ where $f(x), g(x)$ run over $\Q [x]$ and $a(x)$ is the first polynomial and $b(x)$ is the second one in each part of the problem.
        Find $d(x)$ so that $I = (d(x))$ for each part.
    }
    \proof{
        In the proof of Theorem 4.5.7 in the textbook, it is noted that for an ideal $I = \{ f(x) a(x) + g(x) b(x) \mid f(x) , g(x) \in \Q [x] \}$ with fixed $a(x), b(x)$ that $I = (d(x))$ where $(a(x), b(x)) = d(x)$ the greatest common divisor.
        Thus the $d(x)$ that we search for in this problem is given by the answers to the last problem.
    }
\end{exercise}

% hw problem 3 -----------------------------------------------------------------

\begin{exercise}{164}{10}
    \problem{
        Show that the following polynomials are irreducible over the field $F$ indicated. \\\\
        \indent {\bf (a)} $x^2 + 7$ over $\R$ \\
        \indent {\bf (b)} $x^3 - 3x + 3$ over $\Q$ \\
        \indent {\bf (c)} $x^2 + x + 1$ over $\Z _2$ \\
        \indent {\bf (d)} $x^2 + 1$ over $\Z _{19}$ \\
        \indent {\bf (e)} $x^3 - 9$ over $\Z _{13}$ \\
        \indent {\bf (f)} $x^4 + 2x^2 + 2$ over $\Q$ \\
    }
    \proof{ \\\\
        {\bf (a)} Using the bijection between real numbers $x \mapsto \sqrt{7} y $, map $x^2 + 7$ to $(\sqrt{7} y)^2 + 7 = 7y^2 + 7 = 7(y^2 + 1)$.
        Then, by the result in the next problem $\R [y]/(y^2 + 1)$ is a field so $(y^2 + 1)$ is maximal which means $y^2+1$ is irreducible over $\R$.
        This implies that $x^2 + 7$ is irreducible in $\R [x]$. \\\\
        {\bf (b)} For $x^3 - 3x + 3$, note that $p=3$ satisfies the Eisenstein criterion so the polynomial is irreducible in $\Z [x]$.
        Then Gauss' lemma tells us that the same polynomial is irreducible over $\Q$. \\\\
        {\bf (c)} For this one, just plug in the two options to $p(x) = x^2 + x + 1$.
        We have $p(0) = p(1) = 1$ so $p(x)$ is irreducible over $\Z _2$. \\\\
        {\bf (d)} Looking at $x^2 + 1$ over $\Z _{19}$, we know $x^2 + 1$ is irreducible in $\R [x]$ so it is also irreducible in $\Z _{19} [x]$. \\\\
        {\bf (e)} For $x^3 - 9$ over $\Z _{13}$, has no zeros in $\Z _{13}$ so it is irreducible. \\\\
        {\bf (f)} Here we use the Eisenstein criterion with $p=2$ to show that $x^4 + 2x^2 + 2$ is irreducible over $\Z$ and Gauss' lemma tells us that the same polynomial is irreducible over $\Q$. \\\\
    }
\end{exercise}

% hw problem 4 -----------------------------------------------------------------

\begin{exercise}{164}{13}
    \problem{
        Let $\R$ be the field of real numbers and $\C$ that of complex numbers.
        Show that $\R [x] / (x^2 + 1) \cong \C$.
    }
    \proof{
        We will show this with the first isomorphism theorem.
        To do this, we need a surjective homomorphism $\varphi: \R [x] \to \C$ with $\ker \varphi = (x^2 + 1)$.
        Consider the map $\varphi: \R [x] \to \C$ defined by taking the polynomial $a_k x^k + \cdots + a_1 x + a_0$ to the complex number $a_k i^k + \cdots + a_1 i + a_0$. \parspace
        That $\varphi$ is a surjection is immediate by choosing $bx + a \in \R [x]$ for the complex number $a+ib \in \C$.
        Now we verify that $\varphi$ is a homomorphism.
        The addition component can be checked easily.
        For multiplication, pick $a(x), b(x) \in \R [x]$.
        Then we have $\varphi (a_0 + \cdots + a_k x^k) \varphi (b_0 + \cdots + b_l x^l) = (a_0 + \cdots + a_k i^k) (b_0 + \cdots + b_l i^l)$ and $\varphi (a(x) b(x)) = \varphi ( (a_0 + \cdots + a_k x^k) (b_0 + \cdots + b_l x^l) )$ where multplying the polynomials has the same ``structure" as multiplying the complex numbers (ie foiling) so the multiplicative property of the homomorphism $\varphi$ holds. \parspace
        Now to check that $\ker \varphi = (x^2 + 1)$.
        Going to the left, pick any $p(x) \in (x^2 + 1) \implies $ there exists some $f(x) \in \R [x]$ such that $p(x) = f(x) (x^2+1) \implies \varphi (p(x)) = \varphi (f(x) (x^2+1)) = \varphi (f(x)) \varphi (x^2+1) = \varphi (f(x)) (i^2 + 1) = \varphi (f(x)) 0 = 0$ so $p(x) \in \ker \varphi$.
        Now going to the right, fix any $p(x) \in \ker \varphi$.
        Then $\varphi (p(x)) = p_0 + p_1 i + \cdots p_n i^n = 0$.
        Now let $p$ take any complex number as an input instead of just real inputs ($p$ is a member of $\C [z]$ as well as $\R [x]$).
        In the context of $\C$, this means that $p(i) = 0$, which is to say that $z^2 + 1 \mid p(z)$.
        But then all the coefficients are real so we must also have $x^2 + 1 \mid p(x) \implies p(x) \in (x^2 + 1)$. \parspace
        Thus we have all the requirements for $\varphi$ to be the map in the first isomorphism theorem and we can conclude that $\R [x] / (x^2 + 1) \cong \C$.
    }
\end{exercise}

% hw problem 5 -----------------------------------------------------------------

\begin{exercise}{165}{16}
    \problem{
        Let $F = \Z _p$ for some prime number $p$ and $q(x) \in F[x]$ where $q(x)$ is irreducible with degree $n$.
        Show that $F[x] / (q(x))$ has exactly $p^n$ elements.
    }
    \proof{
        I couldn't quite prove equality on this one, maybe I'm missing something super obvious.
        I did prove that the set $F[x] / (q(x))$ has at least $p^n$ elements.
        Since $q(x)$ has degree $n$ we know that $q(x) = a_0 + a_1 x + \cdots + a_n x^n$ with $a_n \neq 0$ and the irreducibility implies that the ideal $(q(x))$ contains only one element (0) that has degreen less than $n$.
        Thus each $p(x)$ with $\deg p(x) < n$ produces a unique element of $F[x] / (q(x))$.
        The polynomial $p(x) \in \Z _p$ has the form $a_0 + a_1 x + \cdots + a_{n-1} x^{n-1}$ where each $a_i$ (of which there are $n$) has $|\Z _p| = p$ options.
        Thus we have at least $p^n$ elements in the set $F[x] / (q(x))$.
        I look forward to reading the solutions to learn how to show the upper bound.
    }
\end{exercise}

% hw problem 6 -----------------------------------------------------------------

\begin{exercise}{171}{6}
    \problem{
        Let $F$ be the field and $\varphi$ an automorphism of $F[x]$ such that $\varphi (a) = a$ for all $a \in F$.
        If $f(x) \in F[x]$, prove that $f(x)$ is irreducible in $F[x]$ if and only if $g(x) = \varphi (f(x))$ is.
    }
    \proof{
        If $\varphi (f(x)) = f(x)$ then this problem is trivial.
        Let's show that this is the case.
        Since $\varphi$ is an automorphism, it is an isomorphism from $F[x] \to F[x]$.
        Fix $f(x) = a_0 + a_1 x + \cdots + a_n x^n \in F[x]$ and consider the following
        \begin{align*}
            \varphi (f(x)) &= \varphi ( a_0 + a_1 x + \cdots + a_n x^n ) \\
            &= \varphi (a_0) + \varphi (a_1 x) + \cdots + \varphi (a_n x^n) \\
            &= \varphi (a_0) + \varphi (a_1) \varphi (x) + \cdots + \varphi (a_n) \varphi (x^n) \\
            &= a_0 + a_1 \varphi (x) + \cdots + a_n \varphi (x) ^n \\
            &= a_0 + a_1 x + \cdots + a_n x^n = f(x)
        \end{align*}
        where made several crucial steps based on the fact that $\varphi$ is a homomorphism with the property that $\varphi (a) = a $ for any $a \in F$.
        Note that all the $a_i \in F$ and $x \in F$ so our steps were justified.
    }
\end{exercise}

\end{document}
